
%!TEX root = Thesis.tex


%%%%%%%%%%%%%%%%%%%%%%%%%%%%%%%%%%%%%%%%%%%%%%%%
%                            Front (Title) page
%%%%%%%%%%%%%%%%%%%%%%%%%%%%%%%%%%%%%%%%%%%%%%%%

%\thispagestyle{empty}
%\vspace*{10mm}
%
%\centerline{DEPARTMENT OF PHYSICS, UNIVERSITY OF JYV\"ASKYL\"A}
%\centerline{RESEARCH REPORT No. ??/2019}
%
%\vspace{25mm} 
%
%\centerline{\bf JET TRANSVERSE MOMENTUM DISTRIBUTIONS }
%\centerline{\bf  FROM RECONSTRUCTED JETS}
%
%\centerline{\bf  IN P--PB COLLISIONS AT \sqrtSnnE{5.02}}
%\centerline{\bf }
%%JET TRANSVERSE MOMENTUM DISTRIBUTIONS FROM RECONSTRUCTED JETS IN P--PB COLLISIONS AT \sqrtSnnE{5.02}
%%Jet Transverse Momentum Distributions From Reconstructed Jets In p--Pb Collisions At $\sqrt{s_{\mathrm{NN}} = 5.02~\mathrm{TeV}$
%
%\vspace{13mm}
%
%
%\centerline{\bf BY}
%\centerline{\bf TOMAS SNELLMAN}
%
%\vspace{13mm}
%
%\centerline{Academic Dissertation}
%\centerline{for the Degree of}
%\centerline{Doctor of Philosophy}
%
%\vspace{13mm}
%
%%\centerline{To be presented, by permission of the}
%%\centerline{Faculty of Mathematics and Natural Sciences}
%%\centerline{of the University of Jyv\"askyl\"a,}
%%\centerline{for public examination in Auditorium FYS 1 of the}
%%\centerline{University of Jyv\"askyl\"a on June 19th, 2019,}
%%\centerline{at 12 o'clock noon}
%
%\centerline{\emph{
%To be presented, by permission of the Faculty of Mathematics and Natural Sciences}}
%\centerline{\emph{
% of the University of Jyv\"askyl\"a, for public examination in Auditorium FYS 1}}
% \centerline{\emph{ of the University of Jyv\"askyl\"a on June 19th, 2019, at 12 o'clock noon}}
%
%\vspace{13mm}
%
%
%
%\centerline{\includegraphics[height=20mm]{Soihtu}}
%
%\centerline{Jyv\"askyl\"a, Finland}
%%\centerline{\today}
%\centerline{June 2019}
%\pagebreak
\thispagestyle{empty}
\section*{Abstract} 
%\addcontentsline{toc}{chapter}{Personal Contribution}
\addtocontents{toc}{\protect\contentsline{chapter}{\protect\numberline{}Abstract}{}{}}

%{\color{red} TODO: Correct format and data}
Snellman, Tomas \\
Jet transverse momentum distributions from reconstructed jets in p--Pb collisions at \sqrtSnnE{5.02}\\
Jyväskylä: University of Jyväskylä, 2019, 140 p. \\
(JYU Dissertations \\
ISSN 2489-9003; 99) \\
ISBN 978-951-39-7800-6 (PDF) \\
~\\

%\section*{Abstract} 
\noindent
In this thesis we study the transverse structure of reconstructed jets via transverse fragmentation momentum, $\jt{}$, distributions in proton-lead (p--Pb) collisions at the centre-of-mass energy per nucleon of 5.02 TeV. The data is measured with the ALICE experiment at the CERN LHC. 

In previous analysis that used two-particle correlations, it has been observed that the measured \jt{} distributions can be factorised into two components, a narrow Gaussian component, and a wide non-Gaussian component. It was argued that these components can be linked to the non-perturbative hadronisation and to the perturbative showering process, respectively. We have observed the same factorisation holds for $\jt{}$ distributions obtained using reconstructed jets. Furthermore although a direct quantitative comparison is not possible, our data is qualitatively compatible with $\jt{}$ distributions measured from two-particle correlations.

Studies of collective flow in high multiplicity p--Pb collisions have seen hints of behaviour that in Pb--Pb collisions has been taken as indication of the creation of Quark Gluon Plasma (QGP), a deconfined state of QCD matter. However studies of jet observables have shown no modification in high multiplicity p--Pb collisions. As expected it has been observed in Pb--Pb collisions that jets traversing through QGP medium will lose energy from interactions with the medium. Thus it remains an open question whether QGP is created in a p--Pb collision. In this thesis we compare measured \jt{} distributions between minimum bias and high multiplicity p--Pb collisions.  Our results show no sign of modification within the current experimental capabilities.

\begin{itemize}[label={},itemindent=-5.5em,leftmargin=5.5em]

\item {\bf Keywords:} jet, jet shape, jet fragmentation, jet reconstruction, heavy ion, p--Pb, transverse momentum, ALICE, CERN, LHC
\end{itemize}
~\\
\newpage

\section*{Personal Contribution} 

\addtocontents{toc}{\protect\contentsline{chapter}{\protect\numberline{}Personal Contribution}{}{}}

The main contributions of the author to the research presented in this thesis are listed below
\begin{itemize}
\renewcommand\labelitemi{--}

\item Jet fragmentation transverse momentum (\jt{}) analysis for  \sqrtSnnE{5.02} p--Pb data
\begin{itemize}
\item Implementation of the observable (\jt{}) for jet analysis
\item Implementation and improvement of the perpendicular cone background method
\item Developing the random background method
\item Performing unfolding of detector effects on the data
\item Systematic uncertainty evaluation
\item Repetition of the analysis with \pythia~and HERWIG Monte Carlo generators and comparison of model results and data
\item Results comparison between high multiplicity and minimum bias p--Pb collisions
\item Comparison of results to \jt{} extracted from two-particle correlations
\end{itemize}
\item Quality assurance (QA) of the Gas Electron Multiplier (GEM) foils that are built into the Time Projection Chamber (TPC) detector of the ALICE experiment. The work was performed at the Helsinki Institute of Physics.
\item At CERN contributing to the maintenance of the level-0 trigger of the electromagnetic calorimeter at the ALICE experiment
\end{itemize} 


%%%%%%%%%%%%%%%%%%%%%%%%%%%%%%%%%%%%%%%%%%%%%%%%
%                            Author page
%%%%%%%%%%%%%%%%%%%%%%%%%%%%%%%%%%%%%%%%%%%%%%%%
\newpage
\thispagestyle{empty}

\vspace*{25mm} 

\begin{table}[h!]
%\centering
  \begin{tabular}{ p{4cm}  p{6cm} }
  {\bf Author} 	& {Tomas Snellman}\\  
  {} 			& {University of Jyv\"askyl\"a}\\  
  {} 			& {Finland}\\  
  {} 			& {}\\  

  {\bf Supervisors} 	& {Dr.~Kim Dong Jo}\\  
  {} 				& {University of Jyv\"askyl\"a}\\  
  {} 				& {Finland}\\  
  {} 				& {}\\  

  {}			 	& {Prof.~Jan Rak}\\  
  {} 				& {University of Jyv\"askyl\"a}\\  
  {} 				& {Finland}\\  
  {} 				& {}\\  

  {\bf Reviewers} 	& {Prof.~Stefan Bathe}\\  
  {} 				& {Baruch College, CUNY}\\  
  {} 				& {USA}\\  
  {} 				& {}\\  

  {}			 	& {Prof.~Marcin Chrząszcz}\\  
  {} 				& {Instytut Fizyki Jądrowej PAN}\\  
  {} 				& {Poland}\\  
  {} 				& {}\\  
    
  {\bf Opponent} 	& {Prof.~Jamie Nagle}\\  
  {} 				& {University of Colorado Boulder}\\  
  {} 				& {USA}\\  
  {} 				& {}\\  

  \end{tabular}
  \label{}
\end{table}
\pagebreak
%%%%%%%%%%%%%%%%%%%%%%%%%%%%%%%%%%%%%%%%%%%%%%%%%
%%                            Dedication page
%%%%%%%%%%%%%%%%%%%%%%%%%%%%%%%%%%%%%%%%%%%%%%%%%
%
%\vspace*{10mm} 
%\thispagestyle{empty}
%\begin{flushright}
%\end{flushright}
%\pagebreak
%%%%%%%%%%%%%%%%%%%%%%%%%%%%%%%%%%%%%%%%%%%%%%%%%
%%                            An empty page
%%%%%%%%%%%%%%%%%%%%%%%%%%%%%%%%%%%%%%%%%%%%%%%%%
%\thispagestyle{empty}
%\mbox{} 
%\pagebreak
%%%%%%%%%%%%%%%%%%%%%%%%%%%%%%%%%%%%%%%%%%%%%%%%
%                           Start page numbering
%%%%%%%%%%%%%%%%%%%%%%%%%%%%%%%%%%%%%%%%%%%%%%%%
%\pagenumbering{roman}
%\setcounter{page}{1}

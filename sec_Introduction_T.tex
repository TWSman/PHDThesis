% !TEX root = thesis.tex

%\section{Introduction}

\begin{abstract}
In this thesis I study harmonic flow coefficients $v_2$ and $v_3$ of identified charged particles in heavy ion collisions. The study is done for Pb-Pb collisions at $\snn = 2.76\tev$ using data simulated by a Multiphase Transport Model (AMPT). The particles used in this study are pions, kaons and protons. Centrality and transverse momentum, $p_T$, dependence of flow coefficients is studied as an average over a large number of events.

Another aspect studied here is the measurement of event-by-event harmonic flow coefficients $v_2$ and $v_3$ and unfolding these distributions. The unfolding method was first studied with a simple Monte Carlo simulation with various magnitudes of flow and multiplicities. The method works well for $v_2$ and for $v_3$ in central collisions but fails for $v_3$ in peripheral collisions. The unfolding method is also applied to the AMPT data.
%Anisotropic flow coefficients $v_2$ and $v_3$ in heavy ion collisions are studied using two methods; an event plane method and two particle correlations. These are tested with a simplified Monte Carlo simulation. The event plane method is also applied to the Multiphase Transport (AMPT) model. Centrality and transverse momentum, $p_T$, dependence of flow coefficients is studied as an average over a large number of events as well as on an event-by-event basis. The $p_T$ dependence of flow coefficients in various centrality bins is compared to ALICE measurements in Pb-Pb collisions at $\snn = 2.76\tev$ in LHC ~\cite{arxiv12055761}. The comparison revealed that for low and very high $p_T$ the model agrees with the data. In the medium-$p_T$ range AMPT fails to describe the data.
\end{abstract}
\tableofcontents

\clearpage
\section{Introduction}
The study of QCD matter at high temperature is of fundamental and broad interest, forming one of the goals of nuclear physics research. Bulk QCD matter and its possible phase transitions between hadronic matter and the quark-gluon plasma (QGP) can be explored in the laboratory, through collisions of heavy atomic nuclei at ultra-relativistic energies. QGP is a state where individual quarks and gluons are no longer confined into bound hadronic states. This corresponds to the conditions in the early universe at the age of $10^{-6}\mathrm{s}$ after the Big Bang.

The deconfinement was first predicted in 1975~\cite{Collins:1975}. The term QGP was introduced by Edward Shuryak in his 1980 review of QCD and the theory of superdense matter~\cite{Shuryak:1980}.

The development of heavy ion physics is strongly connected to the development of particle colliders.  
Experimental study of relativistic heavy ion collisions has been carried out for three decades, beginning with the Bevalac at Lawrence Berkeley National Laboratory (LBNL), and continuing with the AGS at Brookhaven National Laboratory (BNL), CERN SPS, RHIC at BNL and LHC at CERN.

In 1986 the Super Proton Synchrotron (SPS) at CERN started to look for QGP signatures in O+Pb collisions. The center-of-mass energy per colliding nucleon pair $\left(\sqrt{s_{NN}}\right)$ was 19.4 GeV. These experiments did not find any decisive evidence of the existence of QGP. In 1994 a heavier lead (Pb) beam was introduced for new experiments at $\sqrt{s_{NN}}\approx 17\; \mathrm{GeV}$. Although the discovery of a new state of matter was reported, these provided no conclusive evidence of QGP. Now SPS is used with 400 GeV proton beams for fixed-target experiments, such as the SPS Heavy Ion and Neutrino Experiment (SHINE), which tries to search for the critical point of strongly interacting matter.

The Relativistic Heavy Ion Collider (RHIC) at BNL in New York, USA started its  operation in 2000. The top center-of-mass energy per nucleon pair at RHIC, 200 GeV, was reached in the following years. The results from the experiments at RHIC have provided a lot of convincing evidences that QGP was created~\cite{Adcox:2004mh, Adams:2005dq, Arsene:2004fa, Back:2004je}. The newest addition to the group of accelerators capable of heavy-ion physics is the Large Hadron Collider (LHC) at CERN, Switzerland. LHC started operating in November 2009 with proton-proton collisions. First Pb-Pb heavy ion runs started in November 2010 with $\sqrt{s_{NN}}=2.76\; \mathrm{TeV}$,  over ten times higher than at RHIC. Among the six experiments at LHC, the Large Ion Collider Experiment (ALICE) is dedicated to heavy ion physics. Also CMS and ATLAS have active heavy ion programs. 

One of the experimental observables that is sensitive to the properties of QGP is the azimuthal distribution of particles in the plane perpendicular to the beam direction. 
When nuclei collide at non-zero impact parameter (non-central collisions), the geometrical overlap region is asymmetric. This initial spatial asymmetry is converted via multiple collisions into an anisotropic momentum distribution of the produced particles. For low momentum particles ($\pt{} \lesssim 3$ \gevc), this anisotropy is understood to result from hydrodynamically driven flow of the QGP~\cite{Adcox:2004mh, Adams:2005dq, Ollitrault:1992, Heinz:2002, Shuryak:2009}. One way to characterize this anisotropy is with coefficients from a Fourier series parametrization of the azimuthal angle distribution of emitted hadrons. The second order coefficient, which is also known as elliptic flow, shows clear dependence on centrality. The collision geometry is mainly responsible for the elliptic flow. Higher harmonics don't depend that much on centrality. These higher harmonics carry information about the fluctuations in collisions. The event-by-event fluctuations have an increasing importance in measurements. 

In my Bachelor's thesis I  studied methods to determine the event plane and flow coefficients in heavy ion collisions. In this Master's thesis I have performed further analysis on the AMPT data and studied flow coefficients of identified charged particles. One important aspect in flow of different particle species has been quark number scaling. At RHIC energies it was found to work almost perfectly for pions, kaons and protons. This was taken as a strong indication that anisotropic flow at RHIC develops primarily in the partonic phase, and is not strongly influenced by the subsequent hadronic phase~\cite{Lacey:2012ma}. 

At LHC energies  $\sqrt{s_{NN}}=2.76\,\mathrm{GeV}$ it has been observed that in general there is little difference to flow at RHIC energies. The $v_2$ coefficient is about 20\% greater at LHC than at RHIC, depending on the centrality bin. 
The particle identified $v_2$ for kaons and pions follows the same trend.It was
observed~\cite{Lacey:2012ma}, however, that the $v_2$ of proton breaks down the quark number
scaling. So far there is no agreement why this scaling breaks down at LHC or why it works so well at RHIC energies.
%This has been attributed to a blueshift in proton $p_T$ and correcting $p_T$ with a similar redshift restores the scaling. The possible origin of this blueshift is however still unclear.

 In this thesis identified charged particle flow and quark number scaling is studied at LHC energies in the AMPT model. The results are compared to ALICE results.

Another aspect that I studied is the unfolding method. Unfolding is used to restore the original $v_n$ distribution from the observed distribution. In this thesis I use a data-driven unfolding method based on an iterative Bayesian procedure. I first test the performance in a Toy Monte Carlo simulation and later apply it to the AMPT data.

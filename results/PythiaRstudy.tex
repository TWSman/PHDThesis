\documentclass{beamer}
%\usetheme{Berlin}
\usetheme{Copenhagen}
%\usetheme{Dresden}
%\usetheme{Darmstadt}
\usecolortheme{whale}
\useoutertheme{infolines}
\usepackage{listings}
%\usepackage{handoutWithNotes}
%\pgfpagesuselayout{1 on 1 with notes}[a4paper,border shrink=5mm]
\usepackage{sansmathaccent}
\usepackage{verbatim}
\usepackage{etoolbox}
\usepackage{tikz} 
\usetikzlibrary{shapes,arrows}
\usepackage{amsmath}
\usepackage{pgffor}
\usepackage{tikz}
\usetikzlibrary{calc}
\usepackage{hyperref}
\usepackage{anyfontsize}
\pdfmapfile{+sansmathaccent.map}
\title[Jet $j_{T}$]{Jet transverse momentum distribution in p-Pb collisions at ALICE}
%\subtitle{PWG JE presentation}
\author{Tomas Snellman} % (optional, for multiple authors)
\institute{University of Jyv\"askyl\"a, Helsinki Institute of Physics}%
 % (optional)

\include{defs}

\begin{document}

\begin{frame}
\frametitle{$j_T$ signal}
\includegraphics[width=0.90\textwidth]{RcomparisonSignal.pdf} 
\end{frame}

\begin{frame}
\frametitle{RMS}
\begin{columns}
\column{0.5\textwidth}
\includegraphics[width=0.90\textwidth]{RcomparisonRMS.pdf} 
\column{0.5\textwidth}
\begin{itemize}
\item Wide component increases with increasing jet $p_T$ and with increasing $R$ 
\item Narrow component RMS stays constant with increasing jet $p_T$
\item At low jet $p_T$ narrow component has similar behaviour with respect to $R$ as the wide component
\item At high jet $p_T$ narrow component values are reversed for different $R$
\end{itemize}
\end{columns}
\end{frame}

\begin{frame}
\frametitle{Yield}
\begin{columns}
\column{0.5\textwidth}
\includegraphics[width=0.90\textwidth]{RcomparisonYield.pdf} 
\column{0.5\textwidth}
\begin{itemize}
\item Wide component
\end{itemize}
\end{columns}
\end{frame}

\end{document}
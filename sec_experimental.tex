% !TEX root = thesis.tex

\section{Experimental setup and data samples}
\label{sec:experimentaldetails}
The $\sqrtSnnE{5.02}$ $\pPb$ ($1.3 \cdot 10^{8}$ events, $\mathcal{L}_{\mathrm{int}} = \unit[620]{nb^{-1}}$) collisions were recorded in 2013 by the ALICE detector~\cite{aliceDetector}. The details of the performance of the ALICE detector during LHC Run~1 (2009-2013) are presented in Ref.~\cite{alicePerformance}.

The analysis uses charged tracks that are reconstructed with the Inner Tracking System (ITS)~\cite{aliceITS} and the Time Projection Chamber (TPC)~\cite{aliceTPC}. These detectors are located inside the large solenoidal magnet, that provides a homogeneous magnetic field of \unit[0.5]{T}. Tracks within a pseudorapidity range $|\eta| < 0.9$ over the full azimuth can be reconstructed. The ITS is made up of the innermost Silicon Pixel Detector (SPD), the Silicon Drift Detector (SDD) and the outermost Silicon Strip Detector (SSD). Each of these consists of two layers. The TPC is a cylinder filled with gas. Gas is ionised along the path of charged particles. Liberated electrons drift towards the end plates of the cylinder where they are detected. Combining the information from the ITS and the TPC provides a resolution ranging from $1$ to $10\,\%$ for charged particles with momenta from $0.15$ to $\unit[100]{\GeVc}$. For tracks without the ITS information, the momentum resolution is comparable to that of ITS+TPC tracks below transverse momentum $\pt{} = \unit[10]{\GeVc}$, but for higher momenta the resolution reaches $20\,\%$ at $\pt{} = \unit[50]{\GeVc}$~\cite{alicePerformance,aliceBackgroundFluctuation}. 

Neutral particles used in jet reconstruction are reconstructed by the Electromagnetic Calorimeter (EMCAL)~\cite{Cortese:2008zza}. The EMCAL covers an area with a range of $|\eta| < 0.7$  in pseudorapidity and $ 100 \deg $ in azimuth. EMCAL is complimented with the Dijet Calorimeter (DCal)~\cite{DCAL} and Photon Spectrometer (PHOS)~\cite{PHOS} that are situated opposite of the EMCAL in azimuth. PHOS covers 70 degrees in azimuth and $\left| \eta \right| < 0.12$. The DCal is technologically identical to EMCal. The DCal coverage spans over 67 degrees in azimuth, but in pseudorapidity the mid region is occupied by the PHOS. In between PHOS and DCal active volumes, there is a gap of 10 cm. DCal is fully back-to-back with EMCal.

The combination of charged tracks with  $\pt{} > \unit[0.15]{\GeVc}$ and neutral particles with $\pt{} > \unit[0.30]{\GeVc}$ is used to construct jets. 

The V0 detector~\cite{forwarddetectorsTdr} provides the information for event triggering. The V0 detector consists of two scintillator hodoscopes that are located on either side of the interaction point along the beam direction. It covers the pseudorapidity region $-3.7 < \eta < -1.7$ (V0C) and $2.8 < \eta < 5.1$ (V0A). For the 2013 $\pPb$ collisions events are required to have signals in both V0A and V0C. This condition is used later offline to reduce the contamination of the data sample from beam-gas events by using the timing difference of the signal between the two stations~\cite{alicePerformance}.

%For the 2010 $\pp$ collisions, the minimum bias (MB) triggered events are required to have at least one hit from a charged particle traversing the SPD or either side of the V0. 
%The pseudorapidity coverage of the SPD is $|\eta| < 2$ in the first layer and $|\eta| < 1.5$ in the second layer. %Combining this with the acceptance of the V0, the particles are detected in the range $-3.7 < \eta < 5.1$. The minimum %bias trigger definition 

EMCAL is also used to provide the jet trigger used in triggered datasets. EMCAL can be used to trigger on single shower deposits or energy deposits integrated over a larger area. Latter case is used for jet triggers. The EMCAL trigger definition in the 2013 $\pPb$ collisions requires an energy deposit of either \unit[10]{\gev}  for the low threshold trigger or \unit[20]{\gev} for the high threshold trigger in a $32\times32$ patch size.

%For the $\pp$ collisions, similar track cuts as in Ref.~\cite{ALICE:2011ac} are used: at least two hits in the ITS are required, one of which needs to be in the three innermost layers, and 70 hits out of 159 are required in the TPC. In addition, the distance of the closest approach (DCA) of the track to the primary vertex is required to be smaller than $\unit{2}{cm}$ in the beam direction. In the transverse direction, a $\pt{}$ dependent cut DCA $< \unit{0.0105}{cm} + \unit{0.035}{cm} \cdot \pt{}^{-1.1}$ is used, where $\pt{}$ is measured in units of $\GeVc$. These track cuts are tuned to minimize the contamination from secondary particles.

%In $\pPb$ collisions the tracks are selected following the hybrid approach~\cite{hybridExplanation}. In this method tracks with at least one hit in the SPD and at least two hits in the whole ITS are always accepted. In addition, tracks with fewer than two hits in the ITS or no hits in the SPD are accepted, but only if an additional vertex constraint is fulfilled. In addition, the distance of the closest approach (DCA) of the track to the primary vertex is required to be smaller than $\unit{3.2}{cm}$ in the beam direction and smaller than $\unit{2.4}{cm}$ in the transverse direction. This approach is not affected by dead regions ins SPD. Thus it produces an azimuthal angle ($\varphi$) distribution that is as uniform as possible. The momentum resolutions of the two classes of particles are comparable up to $\pt{} \approx 10\;\GeVc$, but after that, tracks without ITS requirements have a worse resolution~\cite{alicePerformance,aliceBackgroundFluctuation}.
In $\pPb$ collisions the tracks are selected following the hybrid approach~\cite{hybridExplanation} which ensures a uniform distribution of tracks as a function of azimuthal angle ($\varphi$). The momentum resolutions of the two classes of particles are comparable up to $\pt{} \approx 10\;\GeVc$, but after that, tracks without ITS requirements have a worse resolution~\cite{alicePerformance,aliceBackgroundFluctuation}.


\section{Experimental Details}
\label{sec:exp}

\subsection{LHC}
\label{sec:lhc}
\subsection{ALICE}
\label{sec:alice}
\subsubsection{Tracking}
\subsubsection{TPC}
\subsubsection{TPC upgrage}
\subsubsection{Particle identification?}
\subsubsection{Calorimeters}
\subsubsection{Forward detectors}
\subsubsection{Muon spectrometer}
\subsubsection{Trigger}

\documentclass{standalone}
\usepackage{tikz}
\usepackage{xcolor}
\usetikzlibrary{shapes,arrows}
\usetikzlibrary{trees}
\usetikzlibrary{shadows.blur}
\usetikzlibrary{positioning}
\usetikzlibrary{decorations.pathmorphing}
\usetikzlibrary{decorations.markings}
\begin{document}
\tikzset{
photon/.style={decorate, decoration={snake}, draw=red},
particlearrow/.style={draw=blue, line width=0.75pt, postaction={decorate},
    decoration={markings,mark=at position .5 with {\arrow[draw=black]{>}}}},
antiparticlearrow/.style={draw=blue, postaction={decorate},
    decoration={markings,mark=at position .5 with {\arrow[draw=black]{>}}}},
particle/.style={draw=blue, line width=0.75pt},
hadron/.style={draw=blue,line width=2pt,postaction={decorate},
    decoration={markings,mark=at position .9 with {\arrow[draw=blue]{>}}}},
antiparticle/.style={draw=blue},
gluon/.style={decorate, draw=orange, line width=0.75pt,
    decoration={coil,amplitude=4pt, segment length=5pt}}
 }
\begin{tikzpicture}
%\draw[step = 4cm, gray, thin] (-3cm,-3cm) grid(8,4cm);

\node[ellipse,draw=orange,fill=orange!20, minimum height=1cm, blur shadow = {shadow blur steps=5},minimum width=2cm] (hard) {};
\coordinate[above=1cm of hard, label=Hard Scattering] (label);
\coordinate[left=1cm of hard] (p1);
\coordinate[above left=1cm and 1cm of hard] (p2);
\coordinate[right=1cm of hard] (p3);
\coordinate[below right=1cm and 1cm of hard] (p4);

\coordinate[above right=1cm and 1.25cm of p3] (vertex1_1);
\coordinate[below right=1cm and 1.25cm of p3]  (vertex1_2);

\coordinate[above right=0.75cm and 1cm of vertex1_1] (vertex2_1);
\coordinate[below right=0.3cm and 1cm of vertex1_1] (vertex2_2);
\coordinate[above right=0.3cm and 1cm of vertex1_2] (vertex2_3);
\coordinate[below right=0.75cm and 1cm of vertex1_2] (vertex2_4);


\coordinate[above right=0.75cm and 1cm of vertex2_1] (vertex3_1);
\coordinate[below right=0.3cm and 1cm of vertex2_1] (vertex3_2);
\coordinate[above right=0.3cm and 1cm of vertex2_2] (vertex3_3);
\coordinate[below right=0.3cm and 1cm of vertex2_2] (vertex3_4);
\coordinate[above right=0.3cm and 1cm of vertex2_3] (vertex3_5);
\coordinate[below right=0.3cm and 1cm of vertex2_3] (vertex3_6);
\coordinate[above right=0.3cm and 1cm of vertex2_4] (vertex3_7);
\coordinate[below right=0.75cm and 1cm of vertex2_4] (vertex3_8);


\draw[particlearrow] (p1) -- (hard);
\draw[particlearrow] (p2) -- (hard);
\draw[particlearrow] (hard) -- (p3);
\draw[particlearrow] (hard) -- (p4);

\draw[particle] (p3) -- (vertex1_2);
\draw[gluon] (p3) -- (vertex1_1);

\draw[particle] (vertex1_1) -- (vertex2_1);
\draw[gluon] (vertex1_1) -- (vertex2_2);
\draw[gluon] (vertex1_2) -- (vertex2_3);
\draw[gluon] (vertex1_2) -- (vertex2_4);

\draw[gluon] (vertex2_1) -- (vertex3_1);
\draw[particle] (vertex2_1) -- (vertex3_2);
\draw[particle] (vertex2_2) -- (vertex3_3);
\draw[particle] (vertex2_2) -- (vertex3_4);
\draw[gluon] (vertex2_3) -- (vertex3_5);
\draw[gluon] (vertex2_3) -- (vertex3_6);
\draw[particle] (vertex2_4) -- (vertex3_7);
\draw[particle] (vertex2_4) -- (vertex3_8);

\node[rectangle,draw=orange, fill=orange!20, below right=-0.5cm and 0cm of vertex3_1,minimum width=1cm, minimum height=6cm,label={below:Hadronisation},blur shadow = {shadow blur steps=5}] (hadr) {};

\coordinate[right=0cm of hadr] (hadron1);
\coordinate[right=2cm of hadron1] (detector1);

\coordinate[above=1.5cm of hadron1] (hadron2);
\coordinate[above=2.5cm of hadron1] (hadron3);
\coordinate[below=1cm of hadron1] (hadron4);
\coordinate[below=2.5cm of hadron1] (hadron5);
\coordinate[right=2cm of hadron2] (detector2);
\coordinate[right=2cm of hadron3] (detector3);
\coordinate[right=2cm of hadron4] (detector4);
\coordinate[right=2cm of hadron5] (detector5);


\draw[hadron] (hadron1) -- node[label=above:Hadrons] {}(detector1);
\draw[hadron] (hadron2) -- (detector2);
\draw[hadron] (hadron3) -- (detector3);
\draw[hadron] (hadron4) -- (detector4);
\draw[hadron] (hadron5) --  (detector5);


\end{tikzpicture}
\end{document}
